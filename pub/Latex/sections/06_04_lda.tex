As we defined programs as being made up sets of behavioral components, and those behaviors are largely unknown to us, we needed a system which would allow us to extract those behaviors in an unsupervised manner. This is actually a well-studied area in languages with NLP approaches in the area called Topic Modeling. Blei \textit{et al.} \cite{blei2003latent} developed a method for unsupervised topic modeling called Latent Dirichlet Allocation and would later expand upon it by developing Hierarchical Latent Dirichlet Allocation (hLDA) \cite{griffiths2004hierarchical}. LDA essentially considers each document to be made up of some weighting of latent topics and each topic to be made up of some weighting of terms. Then, it finds the distribution on these weightings which can be used for inference. hLDA is a hierarchical variant which adds the idea that topics are actually structured in a hierarchy. This approach is more akin to software than flat topics; however, there are not many performant hLDA libraries available for larger datasets. This led to a focus on LDA for the time being. LDA has more performant versions available, such as Microsoft's LightLDA \cite{yuan2015lightlda}, which we utilize in our analyses on the larger dataset.