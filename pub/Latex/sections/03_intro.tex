\IEEEPARstart{T}{here} is a great need to understand how programs behave to ensure the security of software and to create the interpretable building blocks of program behavior by which to expand our understanding of software. We present LIBCAISE, a system that will enable us to gain a more readable understanding of the intrinsic differences between programs. The approach was studied on many commonly used sorting and searching programs as these programs form some of the basic components of many commercially available software, thus providing an excellent platform to develop and test the proposed approaches. LIBCAISE successfully isolates key behavioral components and presents them in a readable way, and maintains this at scale, though it loses some ability to derive fine-grained components it located in the small-scale experiment. Our approach is scalable and extensible towards understanding any software with the possibility to extend the analysis towards malware behavior.

The paper is structured as follows: Section \ref{background} details the background for the subject area, Section \ref{previous_work} discusses the extensive work done towards developing an understanding of program behavior (primarily malware, as most of the work is focused in this area), Section \ref{procedure} details the specifics of our approach, as well as the assumptions made, Section \ref{results} showcases the results of the work, Section \ref{discussion} analyzes the effectiveness of the approach and discusses its limitations, Section \ref{future_work} proposes areas in which the work can be extended, and finally Section \ref{conclusion} summarizes the results of our approach.