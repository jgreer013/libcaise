Software production has been steadily increasing over time, and as such, it is often more cost effective for companies to purchase Commercial Off-The-Shelf (COTS) software to fulfill their needs \cite{agrawal2016trends}. Moreover, the rise of mobile devices and their corresponding app stores has drastically increased the availability of third-party software products/executables over time \cite{statista}. As such, it has become increasingly important to verify software behavior in the context of its expected purpose. 

Normal program verification involves downloading from a trusted source or verifying the md5 hash of the download with the software distributor. This ultimately shifts the responsibility of distributing trustworthy software to the distributor, but there are multiple instances of programs such as malware making their way onto various distribution platforms, such as the Google Play Store \cite{maier2014divide}. 

Often times this is the result of hidden functionality being inserted into a normal program which goes against its intended operational paradigm, such as stealth mining of user data in relatively benign programs \cite{bosu2017collusive} or of cryptocurrency while in use \cite{palmer_2019}. The user (and sometimes the distributor) has no way of knowing what these programs are doing or whether they are operating within their expected behavior profiles. Thus, understanding a program's behavior and how it relates to other programs of a similar class is imperative to ensuring programs are behaving as intended.